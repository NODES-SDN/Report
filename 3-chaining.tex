\section{SDN-based service chaining for mobile networks}

SDN (software-defined networking) has revolutionized designing and managing networks over the past few years. Routers and switches operate with complex software, which is closed source and dedicated to the private network companies. Moreover, each of these devices come with their own configuration interfaces that are different between vendors, which makes the work of network administrators exhausting to configure each of these individual network devices. \cite{FRZ13}

SDN changes the way networks are managed and designed by defining two characteristics. First, SDN detaches the control plane from the data plane. Second, SDN unifies the control plane, thus a single software program (controller) can control numerous data plane elements. The SDN controller has direct access to the data plane through a well-defined API such as OpenFlow. An OpenFlow switch has a packet-handling table, where different rules can be defined in the table. Therefore, based on different rules that match with a subset of traffic, different actions (such as dropping, forwarding and header modification) can be taken. Thus, based on different rules installed on OpenFlow by the controller, an OpenFlow switch can act as a router, firewall, network address translator, switch or something in between. 

As discussed earlier, network instruments are closed and proprietary, which is a barrier to openness. Technical innovation can break down the barrier caused by industry and government in order to breed openness. The future perspective of wireless mobile networks is that any mobile device can connect to any network and move from one network to another seamlessly and freely. Thus, handheld devices can connect to any network regardless of what radio technology it uses and who owns the network. Figure ~\ref{fig:openroads} illustrates an overview of OpenFlow Wireless. \cite{Yap10a}

\begin{figure}[h!t]
\centering
\floatstyle{boxed}
\includegraphics[width=110mm]{"openroad architecture".jpg}
\caption{OpenRoads Architecture \cite{Yap10b}}
\label{fig:openroads}
\end{figure}

Openness and an open architecture can provide improved features in the network delivered by new industry suppliers, which can help the open source community to flourish. OpenFlow is a protocol and also southbound API for the SDN controller, which is installed on top of a router, switch, or access point. Despite the fact that current networking devices such as switches and routers do not have a common external interface, OpenFlow gives the ability to the controller software to control the OpenFlow enabled device. 

Kok-Kiong Yap et al., introduced a mobility manager for the network users who move around. In this method, the mobility manager is aware of every application flow in the network and can choose the route for a specific flow. Therefore, when the user travels from one point to another, the mobility manager becomes aware of the user’s movement and can decide to reroute the flow. Due to the independency of OpenFlow from the physical layer, vertical handoff between various radio networks is seamless and easy. 

To conduct several simultaneous experiments, slicing (or virtualizing) the network is used in order to make a service permit its users to move across multiple physical networks freely. So with slicing, multiple controllers can cooperate together, where each one is responsible for controlling its own slice of network. A slice may comprise one network or many networks; one user or many users; one subset of traffic or all traffic. FlowVisor is an open source application, which can be used to slice OpenFlow networks. 

FlowVisor is an extra layer added between the controllers and the datapath. It slices the network and assigns control of various flows to different controllers. Since FlowVisor communicates to the datapath and controllers using the OpenFlow protocol, datapaths think that they are controlled by a single controller, and the controllers believe that they are controlling their own private network, but actually they are controlling a virtual network. Here the method is to classify the flows based on a policy, which is defined by the network manager, then FlowVisor can determine which flow belongs to each slice, and transmit that to the corresponding controller for that slice. For instance, if controller ‘Z’ is in charge of all John’s traffic, then FlowVisor transmits all John’s traffic to controller ‘Z’. 

Slicing makes experiments in the production network simple. Thus, flowspace and topology define each experiment to be assigned to its own slice, which is implemented by the FlowVisor. Moreover, slicing or network virtualization enables the production network for the versioning, where new features can be introduced in the production slice. Various slices can be assigned with different versions. So in this method, new features can be distributed and tested rapidly, and afterward made available to the whole network, or even shared with other network owners. 

Finally, slicing allows decomposition, so network administrators can decompose a network to further slices with the aid of FlowVisor in order to allocate more flow space to them. Repetition of slicing is logical when there is a hierarchy of control in the network. For instance, the network manager can dedicate a slice to a research group in the computer science department, and then that research group can slice it among different experiments. 

OpenFlow does not have the ability to control the datapath elements, such as enable or disable interfaces, set power levels or assign channels. Controlling the datapath elements is usually done with NetConf, command line interface, or SNMP. This job is difficult when there are numerous slices of network and each slice is suppose to be configured independently. SNMPVisor is a good tool to configure the datapath, which works alongside FlowVisor. In order to slice the configuration, SNMPVisor observes the SNMP control messages, and forwards them to the correct datapath element. However, sometimes it is infeasible to slice the configuration. For instance, assigning different power levels for various slices on current existing WiFi Access Points is impossible. 

\subsection{Pantou and OpenWRT}

Pantou transforms an access point or wireless router into an OpenFlow enabled device. So, basically in Pantou, OpenFlow operates on top of OpenWRT as an application \cite{Yia04}. OpenWRT is an open source extendable operating system for the router, which is fully customizable for the needs of users and developers. It competes with other existing solutions in performance, robustness, extensibility, design and stability. OpenWRT is highly customizable, so it is not intended only for professional high-end users, rather other users who are seeking for high customisability can also benefit from it. \cite{Lor14}

OpenWRT is designed in such a way to be user friendly, therefore users can choose their desired packages, configure them, and build their own custom firmware \cite{Lor14}. Depending on the model of router that the user is using, there are numerous projects for Pantou that can be used to build custom OpenWRT firmware with OpenFlow on top of it. For instance, Backfire, Attitude Adjustment, and Trunk are some of the Pantou projects that can be used to generate firmware. \cite{Sve14}

By employing OpenWRT, wireless freedom can be achieved. OpenWRT, due to its open architecture, enables the user to use features that can be added to OpenWRT. These features are, intrusion detection, stateful packet inspection, and many other features that normally should be paid to physical devices, cost thousands of dollars to do the job effectively. The OpenWRT community goal is to keep this open source firmware generic and always up to date alongside with the advancement of technology. 

\subsection{Network Service Chaining}

The traditional network infrastructure has made network service providers (NSPs) to struggle with the user increment and high traffic demands. While users enjoy decrement of “price per bit”, NSPs are in battle with operational cost (OPEX) and operator investment (CAPEX), which are increasing due to the complexity of current network infrastructure. Current network infrastructure in NSPs, cannot fulfill their needs, and this is because of inflexibility in their network. \cite{Wol13}

Conventionally, advanced services such as, firewall, deep packet inspection (DPI), intrusion detection and prevention systems, caching, etc., are fixed inside the network, which make the network inflexible and static. Moreover, due to this inflexibility, configuration between these network services suffers from lack of automation. Therefore, troubleshooting in the network might consume a lot of time varying from hours to even weeks depending on the size of the network. 

Due to inter-dependencies and the low power of automatic configuration that middleboxes (services) have, if there is a new service to be introduced or removed from the network platform, they need careful engineering and customization. By increasing the automation, the cost of OPEX and CAPEX can be reduced. Decreasing the network touch points, consequently reduces the possible configuration mistakes, which reduces the cost of OPEX. In addition, optimizing and refactoring the use of existing resources by virtualizing certain network functions can reduce the CAPEX cost. 

Today, network operators cannot reuse middleboxes, since they are configured to provide only one service. So, if one box is removed from the network, then the whole chain will break. With current issues that network platforms are facing, the need for dynamic service chaining in SDN is gradually increasing. Dynamic service chaining can bring high availability and rapid failure restoration with reliable testing abilities, to the network platform. 

Figure ~\ref{fig:NSC} illustrates the capabilities of network service chaining (NSC), which can introduce dynamic, upgradable and configurable software to the network. NSC enables data to flow without any interference caused by middleboxes residing at different nodes. Therefore, different services in the network can be utilized only if needed, and can be omitted whenever their service is unnecessary. NSC takes advantage of virtualization, so regardless of the layer (physical or higher layers) where middleboxes are implemented, they can be integrated seamlessly. 
       
\begin{figure}[h!]
\centering
\floatstyle{boxed}
\includegraphics[width=110mm]{"NSC vs current network".jpg}
\caption{NSC vs. Current network service model \cite{Wol13}}
\label{fig:NSC}
\end{figure}

Dynamic NSC, introduces more intelligent traffic steering to the network, which accelerates the performance of the network. When there is a single network domain, middleboxes such as load balancers, traffic schedulers, security gateways, etc., provide better fairness and security. However, when it comes to multiple network domains, further operator investment is needed, either in terms of software or hardware. Thus, this extra operator investment causes serious decline in terms of delay, which depending on the number of policy elements that the data is passing and the cross-domain network load, this deferment varies. However, when NSC is implemented in the edge of a network, multimedia flows and sensitive data can steer through different network domains in predictable and reliable manner. Figure ~\ref{fig:NSCbenefit} depicts this benefit of dynamic NSC. 


\begin{figure}[h!]
\centering
\floatstyle{boxed}
\includegraphics[width=110mm]{"benefits of dynamic nsc".jpg}
\caption{Traffic steering in dynamic NSC \cite{Wol13}}
\label{fig:NSCbenefit}
\end{figure}

\subsection{YANG Model and its usage in OpenDaylight}

YANG is a data modeling language, which is used to model the data for network configuration protocol (NETCONF). YANG uses an XML tree to model hierarchical data, where each node has a name, value or series of child nodes. Moreover, the format of event notifications, which are published by the network elements can be specified by YANG. In the YANG modeling language, the signature of a remote procedure call (RPC) can be specified by data modelers, which can be invoked on network components through the NETCONF protocol. \cite{Bjo10}

In OpenDaylight there are different projects such as Netconf Client (NCC) and Model Driven Service Abstraction Layer (MD-SAL) controller, which use YANG for their modeling language \cite{MDS} \cite{Netconf}. In these projects, YANG permits to model the structure of XML data, to describe the semantic components and their connection, and to model all the elements as a unified system \cite{MDS}.

The YANG data model uses XML, which makes the data self-explanatory. Moreover, it is easy for the data to be utilized in applications and controller components that use controller’s northbound API. The YANG data model facilitates the development of applications and controller components. Therefore, the risk of bad interpretation of the data structure can be reduced through a defined pattern, which creates easier, statically typed API for developing a module with specific functionality. 

\subsection{LISP and its usage in OpenDaylight}

Conventionally, IP addresses operate as both network and device identifier. LISP (Locator/ID Separation Protocol) divides the address functionality into two distinctive roles. Endpoint identifier (EID) identifies a network device in a unique manner and routing locator (RLOC) is the routing address point for endpoint identifiers on the network. One of the advantages of LISP is that EID and RLOC can be utilized in the current IPv4 and IPv6 address structure. In LISP, due to aggregation of EID addresses by more than only one RLOC, scalability in the network deployment is feasible. Moreover, because of this separation in the architecture, device mobility in the network becomes possible, therefore an EID can be moved freely from one point in the network to another by changing the RLOC without requiring modification of the configuration of the device. \cite{OKH14}

Figure \ref{fig:LispPacketForwarding} illustrates the order of sending a packet from the client to the server using the LISP infrastructure. First, the packet is forwarded to the LISP router at the client side, which is called egress tunnel router (ETR), then ETR tries to find the next hop router that can reach the destination EID, by sending a MAP request to the MapServer. Next, MapServer tries to find information binded between EIDs and RLOCs and then replies the routing point address of destination EID to the ETR. ETR encapsulates the packet after receiving the routing point address (RLOC) from the MapServer, and then forwards it to the router that has the corresponding RLOC. The router which has the RLOC, is called ingress tunnel router (ITR), the job of which is to decapsulate the received packet and forward it to the chosen sever. 


\begin{figure}[h!t]
\centering
\floatstyle{boxed}
\includegraphics[width=110mm]{"packet forwarding in lisp".jpg}
\caption{Packet forwarding between LISP routers \cite{OKH14}}
\label{fig:LispPacketForwarding}
\end{figure}

In a network, which is virtualized, EID can play the role of virtual address space and RLOC can play the role of physical network address. In software defined networking, the control plane is separated from the data plane. Therefore, LISP in regards to the data plane, defines how encapsulation of virtualized network addresses should be done in addresses from the underlying network. In addition, the control plane stores the binding information between EIDs and RLOCs, and associated forwarding policies, therefore the data plane can fetch this information whenever it receives new flows. \cite{ODLLISP}  

LISP Flow Mapping project in OpenDaylight utilizes LISP infrastructure, which provides mapping system services. This project, consists of the LISP Map Resolver service and LISP Map Server service in order to store and retrieve the mapping data to ODL applications and also data plane nodes. Mapping data can contain different routing policies including load balancing and traffic engineering. Moreover, it can also contain mapping of virtual addresses (EIDs) to physical network addresses (RLOCs), in order to access virtual nodes. In the LISP mapping service, mappings and policies can be specified through the ODL northbound REST API, in order to utilize this service. In addition, data plane devices, which are compatible with the LISP control protocol can utilize this service via a southbound LISP plugin through the LISP control protocol. Figure \ref{fig:LispInODL} illustrates the above-mentioned components. 

\begin{figure}[h!t]
\centering
\floatstyle{boxed}
\includegraphics[width=110mm]{"lisp in odl".jpg}
\caption{LISP in OpenDaylight \cite{ODLLISP}}
\label{fig:LispInODL}
\end{figure}



\subsection{OpenEPC and nwEPC}

Testbeds are good platforms in order to understand and take advantage of new technologies in short time, and in a realistic operator network environment. Current network functionality is extremely complex and this is because of a wide variety of protocols, components and interfaces that have to work simultaneously, which can impact their behavior. There are numerous challenges in order to have a reliable and cost-efficient testbed that can test applications before the final product development. OpenEPC tries to minimize these challenges by offering a novel approach. \cite{OpenEPC}

OpenEPC entirely simulates the operator core network, by providing a good tool for demonstrations and a profound study of IP communication devices, such as radio access networks, mobile devices and core network up to the service platforms. Due to its openness, its source code can be used to access 3GPP (3rd Generation Partnership Project) standard components. Therefore, developers who are interested in LTE (Long Term Evolution) and EPC (Evolved Packet Core) environments can update these standard components in a shorter period of time. 

OpenEPC is inspired by 3GPP EPC architecture, therefore it provides a realistic testbed by binding different standard radio technologies, where it has absolute control over the operator environment. Moreover, it contains all the elements and main functionalities of 3GPP EPC standards, in addition to its own features, which makes it enables all kinds of IP communications such as LTE, EDGE, HSPA, and even non-3GPP accesses such as WiFi. 

OpenEPC comes with different functionalities, some of which we are going to discuss briefly. Core network mobility management is one of the OpenEPC functionalities, where it has some necessary features for establishing the user plane. These features include the implementation of PMIP (Proxy Mobile IP) and GTP (GPRS Tunneling Protocol) mobility, zero packet loss handovers and multiple APN (Access Point Name) support.  

Integration of 3GPP access networks is another functionality of OpenEPC. It incorporates with RAN (Radio Access Network) components, which makes it enable control over wireless connectivity of devices to connect them to the testbed. Therefore, it makes the experiments have realistic radio conditions with complete support of PS (Packet Switch) and partial support of CS (Circuit Switch). Moreover, aside from cost efficient components that are integrated in OpenEPC, it comes with its own radio emulation nodes, which can be engaged in a totally virtualized environment. 

OpenEPC has a mobility management module along with network ANDSF (Access Network Discovery and Selection Function) on mobile devices, and also takes advantage of location-based handovers and radio conditions, which makes the client choose the suitable access networks in order to prevent packet loss. In addition to above-mentioned functionalities, policy and charging control, harmonized AAA and subscriber management, accounting and charging, distribution and user plane realization are other features of OpenEPC. 

OpenEPC is highly modular with configurable architecture, which makes it suitable for any research development in the field of applications, networks and services. It can simulate a number of  different deployment scenarios, which varies from the core network in a single box, to medium-scale distributed testbeds. Figure \ref{fig:OpenEPC} depicts how OpenEPC virtualizes a real network operator environment.  

\begin{figure}[h!t]
\centering
\floatstyle{boxed}
\includegraphics[width=110mm]{"OpenEPC".jpg}
\caption{OpenEPC Testbed \cite{OpenEPC}}
\label{fig:OpenEPC}
\end{figure}

The nw-EPC is an open source software package, which is an implementation of the SAE (System Architecture Evolution) gateway. It is written completely in the C programming language and runs as a single-threaded UNIX process, where the data plane and control plane are both running in the same address space. The goal of nw-EPC is to set up a framework, which implements LTE SGW (Serving Gateway) and PGW (Packet Data Network Gateway) \cite{nwEPC}. Figure \ref{fig:SAE} illustrates how in SAE, different technologies such as 3GPP WLAN, GPRS, Evolved RAN, and non-3GPP networks are transparently unified \cite{SAE}.

\begin{figure}[h!t]
\centering
\floatstyle{boxed}
\includegraphics[width=110mm]{"SAE".jpg}
\caption{System Architecture Evolution \cite{SAE}}
\label{fig:SAE}
\end{figure}

The control plane in nw-EPC incorporates the SAE gateway behaviour with the help of expandable FSM framework. This enables the nw-EPC node gateway to function as both SAE, and PGW gateways and also other gateway functions, such as WiMAX ASN-GW or GGSN. The data plane in nw-EPC supports IPv4 packet classification, and basic GTP (GPRS Tunneling Protocol) tunneling/detunneling. However, there is lack of support for IPv6, DPI, QOS, traffic shaping and policing.  

\clearpage
